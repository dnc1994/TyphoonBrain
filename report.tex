\documentclass[12pt]{article}
\usepackage[top=1.10in, bottom=0.85in, left=1.15in, right=1.15in]{geometry}
\linespread{1.3}
\usepackage{ctex}
\usepackage[colorlinks, citecolor=green, linkcolor=blue, menucolor=red, CJKbookmarks=true]{hyperref}
\usepackage{fancyhdr}
\usepackage{extramarks}
\usepackage{titling}
\usepackage{indentfirst}
\usepackage{amsmath}
\usepackage{graphicx}
\usepackage{array}
\usepackage{bibentry}
\usepackage{natbib}

\iffalse
\AtBeginDocument{
\begin{CJK*}{GBK}{SimSun}
\CJKindent
\sloppy\CJKspace
\CJKtilde
}
\AtEndDocument{\end{CJK*}}
\fi

\begin{document}

\pagestyle{fancy}
\lhead{\textbf{{\thetitle}}}
\rhead{\textbf{\nouppercase{\firstleftmark}}}
\cfoot{\thepage}

\title{\textbf{基于神经网络的台风对直接经济损失的决策模型}}
\author{章凌豪 13307130225\\ 李斯哲 13307130370 \\ 施\hspace{12pt}璇 13307130467}
\date{}
\maketitle

{\bf 摘要:}台风是自然界最危险的灾害之一,同时也是造成直接经济损失最多的自然灾害之一。本文采用神经网络模型与BP算法结合的形式建立了一个对于我国台风灾害拥有普适性的直接经济损失评估模型。并分析了这一模型的稳定性,可靠性以及误差来源,最后尝试将这一模型推广到其他类型的评估之中,以获得更为广泛的适用性。

\tableofcontents

\clearpage

\section{课题背景}

随着全球自然灾害风险日趋严重,灾害造成的损失亦趋向复杂化,自然灾害损失评估逐渐成为科研中一个愈来愈重要的方向。而台风作为造成经济损失最大的自然灾害之一,也越来越受到专家们的重视。我国又是台风灾害高发国家,每年因台风造成的直接经济损失高达290.5亿元。建立一个拥有科学性,平衡性,适合我国国情的评估模型能够与应急规划、救灾过程充分结合来获得最佳的应用效果,这对于中国来说是非常必要的。\\
\indent 在之前的开题报告中已经提到目前国内外常见的几个比较合理种模型,如应用回归分析,模糊数学等。而小组拟采用神经网络模型来建立一种普适的台风导致直接经济损失评估模型,进行分析选取合适的台风数据,并结合BP算法对误差进行分析和对模型进行修正,以得到较为准确的适合我国的评估模型。最后以某一沿海受灾省市为数据资源,对模型进行实用性分析,测试该模型的可靠性。

\vspace{20pt}

\section{符号约定}

在计算台风造成的直接经济损失时,可利用损失率进行估算,即$D=Td$,其中D为直接经济损失,d为损失率。\\
\indent 影响经济损失率的有台风过程中受灾地区的最低气压、台风最大风速、天文大潮指数、降雨量极值、台风持续时间、影响范围等自然因素以及预警报时间、受灾地区GDP、耕地面积、区域人口、财产新旧程度等人为影响。\\
\indent 我们所建立的模型选取了以下影响因素:
\begin{table}[h]
\begin{tabular}{|l|l|}
\hline
x1	&	受灾地区最低气压				\\ \hline
x2	&	台风最大风速					\\ \hline
x3	&	降雨量极值						\\ \hline
x4	&	滞留时间						\\ \hline
x5	&	(登录时间) - (预警报时间)		\\ \hline
x6	&	区域人口占该省市人口比例		\\ \hline
x7	&	\begin{tabular}[c]{@{}l@{}}(该省市耕地面积) *\\ (台风影响范围占该省市总面积比例)\end{tabular}	\\ \hline
x8	&	\begin{tabular}[c]{@{}l@{}}(该省市GDP) *\\ (台风影响范围占该省市总面积比例)\end{tabular}		\\ \hline
\end{tabular}
\end{table}

\indent 变量及参数说明:
\begin{table}[h]
\begin{tabular}{|l|l|}
\hline
$x_i$		&	输入向量 i = 1, 2, ..., 8				\\ \hline
$w_{ij}$	&	$x_i$与隐藏层节点值$hi_j$的连接权值		\\ \hline
$hi_j$		&	隐藏层输入向量元素						\\ \hline
$ho_j$		&	隐藏层输出向量元素						\\ \hline
$b_j$		&	隐藏层神经元阈值						\\ \hline
$\theta_j$	&	$ho_j$与节点值$yi$的连接权值			\\ \hline
$yi$		&	输出层输入向量元素						\\ \hline
$yo$		&	实际输出								\\ \hline
$b$			&	输出层神经元阈值						\\ \hline
$d$			&	期望输出(实际经济损失率)				\\ \hline
\end{tabular}
\end{table}

\indent 由于输入层为8个节点,输出1个节点,因此可设定3个隐藏层节点数目。由此得到反向传播(Back Propagation)模型:

\begin{itemize}
\item 输入向量 —— $X = (x_1, x_2, ..., x_8)$
\item 输出层输入 —— $yi$
\item 输出层输出 —— $yo$
\item 隐藏层输入 —— $hi = (hi_1, hi_2, hi_3)$
\item 隐藏层输出 —— $ho = (ho_1, ho_2, ho_3)$
\end{itemize}

\clearpage

\section{模型建立}

\subsection{模型假设}

\begin{enumerate}
\item 模型中台风灾害造成的直接经济损失只与选取的自然及人为因素有关,不包含偶然性;
\item 经济损失值只与台风最大风速,最低气压,受灾地区降雨量极值,台风滞留时间,预警报时间,受灾地区人口,耕地面积和GDP相关,其他因素影响较小,该模型中忽略不计;
\item 忽略台风未登陆该受灾地区的影响,滞留时间由登陆的完整时间计算得到,不考虑台风离境后的状况;
\item 假设其他自然因素如暴雨等由台风影响,其对经济造成的损失包含在改模型中;
\end{enumerate}

\subsection{基本设定}

设输入函数为$F = x_1 w_1 + x_2 w_2 + ... + x_8 w_8$;激活函数为$y = f(F) = \frac{1}{1 + e^{-F}}$。\\

\indent 对于一个给定的样本输入:\\

\begin{align}
hi_j &= \sum_{i=1}^8 w_{ij} x_i - b_j, j = 1, 2, 3\\
ho_j &= f(hi_j), j = 1, 2, 3\\
yi &= \sum_{i=1}^3 \theta_j ho_j - b, j = 1, 2, 3\\
yo &= f(yi), j = 1, 2, 3
\end{align}

\indent 下面是一些基本的参数设置:

\begin{itemize}
\item 根据实际输出$yo$和期望输出$d$,计算误差$u=\frac{1}{2}(d - yo)^2$
\item 设定允许最大误差为$\epsilon$
\item 设定学习速率为$\eta$
\item 设定最大学习次数为$M$
\end{itemize}

\subsection{求解偏导}

分别求解误差函数对于输出层和隐藏层节点值的偏导。\\

\begin{align}
-\delta_y &= \frac{\partial u}{\partial (\theta)_j} = \frac{\partial u}{\partial yi} \frac{\partial yi}{\partial (\theta)_j} \nonumber \\
&= \frac{\partial (\frac{1}{2}(d - yo)^2)}{\partial yi} \frac{\partial (\sum_{j=1}^3 (\theta)_j ho_j - b)}{\partial (\theta)_j} \nonumber \\
&= -(d - yo) f'(yi) ho_j
\end{align}

\begin{align}
-\delta_hj &= \frac{\partial u}{\partial hi_j} = \frac{\partial (\frac{1}{2}(d - yo)^2)}{\partial ho_j} \frac{\partial ho_j}{\partial hi_j} \nonumber \\
&= -\delta_y \theta_j f'(hi_j)
\end{align}

\subsection{修正参数}

对参数$w_{ij}$和$\theta_j$进行修正。\\

\begin{align}
\Delta w_{ij} &= - \mu \frac{\partial u}{\partial w_{ij}} = \mu \delta_j x_i\\
w_{ij} &= w_{ij} + \eta \Delta w_{ij} = w_{ij} + \mu \eta \delta_j x_i
\end{align}

\begin{align}
\Delta \theta_j &= - \mu \frac{\partial u}{\partial \theta_j} = \mu \delta_y ho_j\\
\theta_j &= \theta_j + \mu \eta \delta_y ho_j
\end{align}

\indent 重新计算误差$u = \frac{1}{2}(d - yo)^2$,直到$u < \epsilon$,循环步骤,选取新的样本。\\
\indent 另外为了简化模型,可令$w_{ij1} = w_{ij2}, j1, j2 = 1, 2, 3$。

\clearpage

\section{参数拟合}

\subsection{拟合数据}

我们搜集了19组近年来的台风数据,对建好的神经网络进行训练,结果如下:

\begin{table}[h]
\begin{tabular}{|l|l|l|l|l|l|l|l|l|l|l|}
\hline
{\bf \#} & $x_1$ & $x_2$ & $x_3$ & $x_4$ & $x_5$ & $x_6$ & $x_7$ & $x_8$ & $d$ & $yo$ \\ \hline 
1 & 975.0 & 33.0 & 505.0 & 20.0 & 57.0 & 0.72 & 88.9 & 17320.1 & 17.62 & 23.46 \\ \hline
2 & 910.0 & 60.0 & 573.0 & 10.0 & 68.0 & 0.9 & 65.68 & 3150.65 & 119.53 & 133.99 \\ \hline
3 & 945.0 & 50.0 & 444.0 & 18.0 & 83.0 & 0.67 & 297.01 & 10500.88 & 56.46 & 59.13 \\ \hline
4 & 960.0 & 40.0 & 646.0 & 3.0 & 23.0 & 0.85 & 62.03 & 2975.0 & 57.874 & 56.36 \\ \hline
5 & 960.0 & 40.0 & 250.0 & 3.0 & 26.0 & 0.29 & 73.43 & 19659.7 & 81.5 & 83.6 \\ \hline
6 & 985.0 & 28.0 & 146.0 & 16.0 & 24.0 & 0.17 & 34.0 & 6826.01 & 8.79 & 8.33 \\ \hline
7 & 940.0 & 45.0 & 320.0 & 18.0 & 74.0 & 0.37 & 93.68 & 23115.38 & 177.0 & 162.44 \\ \hline
8 & 958.0 & 35.0 & 466.0 & 13.0 & 33.0 & 0.76 & 93.84 & 16619.68 & 27.95 & 34.03 \\ \hline
9 & 955.0 & 42.0 & 923.4 & 20.0 & 44.0 & 0.78 & 197.49 & 48729.7 & 134.46 & 157.92 \\ \hline
10 & 955.0 & 42.0 & 1014.0 & 7.0 & 67.0 & 0.82 & 162.9 & 32925.5 & 124.0 & 128.49 \\ \hline
11 & 955.0 & 42.0 & 250.0 & 7.0 & 67.0 & 0.18 & 22.23 & 3926.24 & 19.06 & 13.36 \\ \hline
12 & 955.0 & 45.0 & 450.0 & 20.0 & 39.0 & 0.8 & 160.0 & 32122.0 & 236.3 & 197.83 \\ \hline
13 & 955.0 & 40.0 & 170.0 & 10.0 & 30.0 & 0.24 & 60.7 & 13696.1 & 10.76 & 13.18 \\ \hline
14 & 950.0 & 42.0 & 1800.0 & 21.0 & 41.0 & 0.6 & 55.2 & 16800.0 & 10.0 & 18.94 \\ \hline
15 & 985.0 & 25.0 & 260.0 & 15.0 & 68.0 & 0.66 & 81.5 & 13002.7 & 10.16 & 13.91 \\ \hline
16 & 975.0 & 35.0 & 170.0 & 4.0 & 35.0 & 1.0 & 37.9 & 1785.42 & 5.0 & 3.75 \\ \hline
17 & 970.0 & 35.0 & 273.0 & 14.0 & 30.0 & 0.17 & 43.04 & 9690.0 & 19.8 & 23.15 \\ \hline
18 & 960.0 & 42.0 & 825.0 & 10.0 & 44.0 & 0.9 & 65.68 & 2250.0 & 58.13 & 51.36 \\ \hline
19 & 990.0 & 23.0 & 166.0 & 12.0 & 64.0 & 0.52 & 64.2 & 9100.0 & 5.0 & 4.01 \\ \hline
\end{tabular}
\end{table}

设置$\eta = 0.01$,神经网络基本在300\~400次训练后就收敛了。误差结果如下:

\begin{itemize}
\item 总误差为$u = \sum_{i=1}^{19} \frac{1}{2}(d_i - yo_i)^2 = 1374.3$
\item 均方误差为$\sigma = \sqrt{\frac{\sum_{i=1}^{19}(d_i - yo_i)^2}{19}} = 12.03$
\end{itemize}

\subsection{误差分析}

可以看到,模型的拟合结果还是不错的,这说明所选取的影响因素对经济损失的解释力是不错的。但我们也看到,对于部分数据,实际输出与期望输出相差较远,尤其是第12组数据。这些误差可能源于以下原因:

\begin{enumerate}
\item 隐藏层节点数目不够,刻画影响因素与结果之间关系的能力不够。但是要指出,对于8个解释能力没有太大差距的输入特征来说,3个隐藏层节点的数目正合适。增加隐藏层节点或许能使神经网络在训练数据上表现更好,但存在过拟合的风险。而且增加节点数也会减慢训练的速度。
\item 数据本身存在一定偏差。对于这个问题可以通过去掉离群数据或者对数据进行正规化来解决。但这里其实没有一组数据是特别明显的离群数据,所以这么做不合适;我们曾尝试对数据做正规化,但似乎改进并不显著。
\item 输入特征选取不够合理。这是最有可能的原因,不过从数据的收集来看,如果要加入其它特征,可能会遇到难以找到充足数据的问题。所以可以考虑设计进一步的验证,利用AutoEncoder之类的模型去掉一些特征,使模型的预测能力更加准确。
\end{enumerate}

总体上来讲,作为一个短时决策模型来讲,我们建立的神经网络是不错的。虽然准确性还有待提高,但它能在像第14组数据这种有一个或几个特征的数值特别大的情况下不被误导,大致预测出台风的危害等级,对于抗灾指挥是有指导意义的。

\vspace{20pt}

\section{模型分析}

\subsection{模型优点}

\begin{enumerate}
\item 散化复杂的影响因子。BP神经网络作为模式识别强有力的工具,适合影响因子较多的模型。台风灾害造成的经济损失预测有多个影响因素,确定多个因素及其影响函数较为困难。因此采用BP模型,将影响因子散化可以学弱由影响函数的不确定性带来的误差。
\item 非线性映射。BP神经网络实质上实现了一个从输入到输出的映射功能,数学理论证明三层的神经网络就能够以任意精度逼近任何非线性连续函数。这使得该明星非常适合求解类似损失预估这类内部机制复杂的问题。
\item 容错能力强。由于模型中的影响因子被散化,并且采用带有自学习的模式,其系统中的某一个或一部分(占比例较小)的神经元受到破坏后仍然可以正常工作,整个系统的容错能力较强。

\end{enumerate}

\subsection{模型缺点}

\begin{enumerate}
\item 激活函数难以选择。由于神经元各层的激活函数并非由输入神经元直接映射到输出神经元,因此激活函数依赖的关系过于抽象,难以选择。
\item 收敛速度慢,样本要求高。BP神经网络算法实质上为梯度下降法,但是经济损失模型较为复杂,导致其算法低效,收敛速度慢。因此,在模型的检测中需要大量函数来修正神经元间的连接权重。但是由于台风这一现象的过程较为复杂,人为确定的影响因子相关值偏差很大,使得选取大量典型样本成为一个很困难的问题。

\end{enumerate}

\vspace{20pt}

\section{模型推广}

前文所建立的模型对于台风拥有比较好的预测及评估效果,而由于所使用的神经网络模型及BP算法并没有对于台风的特异性,因此同样可以将模型推广至其他的自然灾害,如地震,海啸,火山等的经济损失评估。但推广的同时需要根据所评估的自然灾害种类来改变需要使用到的影响因子,以及调整输入函数和激活函数,已达到对于所评估的自然灾害有比较好的预测效果。\\
\indent 进一步推广,这一模型不仅可以用于评估自然灾害的经济损失,也可以用于其他的评价问题,如对于旅游景点的综合评估,国家综合实力的评估等等。因此此模型在评估模型之中拥有较为广泛的适用性,可作为一种常用的数学评估模型。

\end{document}
